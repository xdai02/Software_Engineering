\chapter{软件质量}

\section{软件质量}

\subsection{软件质量(Software Quality)}

一个质量好的软件从一个优秀的团队和开发过程开始,并且正确地使用系统架构和设计模式,后期严格的审查和测试都能够有效地增加软件的质量。\\

软件质量的特征包括:

\begin{itemize}
    \item 功能性
    \item 可靠性
    \item 易使用性
    \item 效率
    \item 可维护性
    \item 可移植性
\end{itemize}

客户/用户而言,他们的满足度取决于软件的质量、产品的一致性和是否在预算和预期时间内交付。\\

一个高质量的软件应该符合以下特征:

\begin{itemize}
    \item 易于使用
    \item 保证安全和隐私
    \item 较少的维护成本
    \item 较少的bug
    \item 客户满意度所带来的收益
    \item 在未来的产品中可以复用技术
\end{itemize}

\vspace{0.5cm}

\subsection{软件质量保证(SQA/QA, Software Quality Assurance)}

一个公司通常有一个独立的SQA小组与开发团队合作,评估正在生产的产品质量。即使在软件开发过程中注重错误的预防,也不能减少对QA的需求。\\

SQA小组的职责主要包括:

\begin{itemize}
    \item 准备SQA计划:评估标准、审查、错误跟踪。
    \item 协助开发过程:分析软件的缺陷和流程。
\end{itemize}

\newpage

\section{缺陷预防}

\subsection{代码评审(Code Review)}

软件缺陷很大一部分是来自于对需求的定义和理解不正确,code review是最有效的去除缺陷的手段。越早去除这些潜在的缺陷,所需的代价也越小。\\

Code review不是为了去刻意批斗某个开发者,而是为了团队成员之间相互了解学习,加深成员对系统的理解,使团队成员的代码更加健壮,提早发现代码缺陷。\\

Code review由一组技术人员组成,主要是为了发现软件中功能和逻辑的漏洞,验证功能是否满足了客户/用户的需求,保持项目的可维护性。\\

在code review过程中,可以根据检查清单(checklist)对产品进行检查,并记录下存在的问题。Code review的目的在于发现问题,而不是解决问题。\\

Code review的好处包括:

\begin{itemize}
    \item 提升系统的可维护性

    \item 及早发现潜在bug,降低事故成本。

    \item 促进团队内部知识共享,提高团队整体水平。

    \item 对于评审人员来说是一种思路重构的过程,可以帮助更多的人理解系统。

    \item 彼此能熟悉对方模块。
\end{itemize}

\newpage

\section{缺陷检测}

\subsection{缺陷检测(Defect Detection}

缺陷检测包括三个过程:

\begin{enumerate}
    \item 测试(testing):根据测试用例发现错误。
    \item 调试(debugging):查找并消除故障的原因。
    \item 监控(monitoring):监视有关状态和行为的信息。
\end{enumerate}

缺陷检测可以通过静态分析(static analysis)和动态分析(dynamic analysis)两种方式进行。\\

静态分析包括code review、向他人解释代码流程、借助自动化的工具检查语法语义错误及代码规范。\\

动态分析包括:

\begin{itemize}
    \item 黑盒测试(black-box testing):测试子系统的输入/输出行为。
    \item 白盒测试(white-box testing):测试子系统或类的内部逻辑。
\end{itemize}

\vspace{0.5cm}

\subsection{测试}

测试最好是由非开发软件的人进行,因为开发者一般会更加注重能够使程序正常工作的数据。当其他人使用程序时,往往会发现程序的问题。\\

因此一个测试人员非常有必要对系统由足够的了解,掌握各种测试方法和技术。\\

测试分为4种类型:

\begin{enumerate}
    \item 单元测试(unit testing):对每个类/子系统单独测试,确保每个模块的正确性。

    \item 集成测试(integration testing):将模块组装成子系统,确保各个模块连接在一起后也能正常工作。

    \item 系统测试(system testing):对整个系统进行测试。

    \item 验收测试(acceptance test):由客户/用户进行测试。
\end{enumerate}

\newpage