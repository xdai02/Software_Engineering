\chapter{UML}

\section{UML}

\subsection{UML(Unified Modeling Language)}

统一建模语言UML是一种为面向对象系统的产品进行说明、可视化和编制文档的一种标准语言。\\

UML中包含了一系列不同类型的图:

\begin{itemize}
    \item 类图
    \item 组件图
    \item 部署图
    \item 对象图
    \item 封装图
    \item 复合结构图
    \item 剖面图
    \item 用例图
    \item 活动图
    \item 状态机图
    \item 序列图
    \item 通讯图
    \item 交互概览图
    \item 时序图
\end{itemize}

这些图主要可分为三大类:

\begin{enumerate}
    \item 功能模型(functional model):从用户的角度展示系统的功能,如用例图。
    \item 对象模型(object model):采用对象、属性、操作、关联等展示系统的结构,如类图、对象图。
    \item 动态模型(dynamic model):展现系统的内部行为,如时序图、活动图、状态图。
\end{enumerate}

\newpage

\section{用例图}

\subsection{用例图(Use Case Diagram)}

用例图用于描述从用户角度所看到的系统功能。通过用例图,人们可以获知系统不同种类的用户和用例。\\

\begin{figure}[H]
    \centering
    \begin{tikzpicture}
        \draw (0,0) rectangle (6,17);
        \node at (1.5,16.5) {Bank ATM};
        \draw (3,14.5) ellipse (2 and 1) node {查看余额};
        \draw (3,12) ellipse (2 and 1) node {存款};
        \draw (3,9.5) ellipse (2 and 1) node {取款};
        \draw (3,7) ellipse (2 and 1) node {转账};
        \draw (3,4.5) ellipse (2 and 1) node {维护};
        \draw (3,2) ellipse (2 and 1) node {维修};

        \draw (-3,13) circle (0.25);
        \draw (-3.5,12.5) -- (-2.5,12.5);
        \draw (-3,12.75) -- (-3,12);
        \draw (-3,12) -- (-3.5,11.5);
        \draw (-3,12) -- (-2.5,11.5);
        \node at (-3,11) {客户};

        \draw (-3,6) circle (0.25);
        \draw (-3.5,5.5) -- (-2.5,5.5);
        \draw (-3,5.75) -- (-3,5);
        \draw (-3,5) -- (-3.5,4.5);
        \draw (-3,5) -- (-2.5,4.5);
        \node at (-3,4) {技术人员};

        \draw (9,8.5) circle (0.25);
        \draw (9.5,8) -- (8.5,8);
        \draw (9,8.25) -- (9,7.5);
        \draw (9,7.5) -- (9.5,7);
        \draw (9,7.5) -- (8.5,7);
        \node at (9,6.5) {银行};

        \draw (-2,12.5) -- (1,14.5);
        \draw (-2,12.5) -- (1,12);
        \draw (-2,12.5) -- (1,9.5);
        \draw (-2,12.5) -- (1,7);

        \draw (-2,5.5) -- (1,4.5);
        \draw (-2,5.5) -- (1,2);

        \draw (8,8.5) -- (5,14.5);
        \draw (8,8.5) -- (5,12);
        \draw (8,8.5) -- (5,9.5);
        \draw (8,8.5) -- (5,7);
        \draw (8,8.5) -- (5,4.5);
        \draw (8,8.5) -- (5,2);
    \end{tikzpicture}
    \caption{银行ATM系统}
\end{figure}

\newpage

\section{类图}

\subsection{类图(Class Diagram)}

类图用于显示模型的静态结构,例如类的内部结构以及类与类之间的关系。\\

一个类包含属性和方法两部分,在类图中公共的属性和方法使用“+”表示,私有的属性和方法使用“-”表示。\\

\begin{figure}[H]
    \centering
    \begin{tikzpicture}
        \begin{class}[text width = 6cm]{Student}{-7,-4}
            \attribute{- name : String}
            \attribute{- gender : String}
            \attribute{- major : String}
            \operation{+ Student(String name, String gender, String major)}
            \operation{+ getName() : String}
            \operation{+ getGender() : String}
            \operation{+ getMajor() : String}
        \end{class}
    \end{tikzpicture}
\end{figure}